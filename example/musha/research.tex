\chapter{関連研究}
\section{マルチFPGAシステムでのボード間通信}
有名なマルチFPGAシステムとしては、たとえば、Microsoft社の提供するBing検索エンジンに利用されている
Catapult\cite{Catapult}がある。Catapultでは、ボード間の接続にSerial Attached SCSIの物理層を用いた
独自のシリアルI/Oを装備している。
また、EthernetやPCIeを使う例としては、Berkeley Univ.のBEE3\cite{BEE3}、EthernetとInfinibandを使う例として、
Imperial CollegeのAXCEL\cite{Axcel}がある。国内の例では、たとえば、東北大学の密結合クラスタ\cite{Sano}も
Ethernetの物理層を利用している

これらのマルチFPGAシステムはすべてパケット交換方式によってボード間の通信を行っている点で、
回線交換方式のFiCシステムとは異なっている。
%マルチFPGAシステムのボード間接続は、EthernetなInfinibandなどの標準インタフェースを
%用いる例が多い。Berkeley大学のBEE3\cite{BEE3}は、ボード上のFPGAはパラレルI/Oを用いているが
%ボード間の接続はEthernetやPCIeを用いており、Imperial CollegeのAXCEL\cite{Axcel}も
%FPGAボード間の
%接続は、EthernetとInfinibandであり、
%東北大学の密結合クラスタ\cite{Sano}もEthernetの物理層を利用している。独自の
%シリアルI/Oを用いたシステムとしてはMicrosoftのCatapult\cite{Catapult}があり、
%Serial Attached SCSIの物理層を利用している。これらのすべてのマルチFPGAシステムでは
%パケット転送を用いており、STDMを用いているCiFとは全く異ったネットワークである。

\section{大規模畳み込みニューラルネットワーク}
今回の検証ではAkexNet\cite{alexnet}をベンチマークとして使用したが、近年
注目されている比較的小規模な畳み込み層を何層も重ねた形の
GoogLeNet\cite{googlenet}やResNet\cite{resnet}についても検討しなければ
ならない。一般的には深いネットワークのほうがパラメータ数が少なくなる傾
向があり、この期もその傾向が続く場合はハードウェアデザインに大きな影響
を与える可能性がある。ただし、これに関しては反論\cite{dodeepnet}もある。

\section{FPGAベースのCNNアクセラレータ}
CNN、もしくはより汎用的なニューラルネットワークの処理を大規模なシステ
ムによって高速化、省電力化する代表的な研究としてDaDianNaoマシ
ンラーニングスーパーコンピュータ\cite{dadiannao}が挙げられる。
これは推論学習のミリ秒単位での高速化に焦点を当てたプロジェクトで、重みデータを可能な限り
ローカルに保存できるような設計をしている。それによってメモリ帯域幅のボトルネックを
改善するというコンセプトではFiCプロジェクトとも関連が強い。
%これは識別学習のミリ秒単位での高速化に焦点をあてたプロジェクトで、重みデータを可能な限りローカルに保存できるような設計をし、メモリバンド幅のボトルネックを解消するというコンセプトでは我々のプロジェクトとも関連性が強い。
ただし、\cite{dadiannao}はASICチップシステムであり、FPGA-GPUシステムを用いる点では異なっている。



CNNに対するFPGAを用いた高速化の最初期の研究としては、2009年のCNP \cite{cnp}が挙げられる。
これはソフトウェア実行を中心としながら、畳み込み演算のフィルタリング処理をハードウェアで高速化したものである。
2010年にはFPGA上で完全なCNN実装\cite{NEC_Labs_America1}\cite{NEC_Labs_America2}\cite{NEC_Labs_America3}が行えるようになった。

現在、FPGAベースのCNNアクセラレータはUCLAのマルチレイヤーCNNアクセラレータ\cite{fpgaopt}が有名である。
今回実装を行ったアクセラレータの演算部はこの設計を参考にした。

他にも、Microsoft社のCatapultのFPGAで動作するCNNアクセラレータ\cite{ms_fpgacnn}が挙げられる。
このCatapultのアクセラレータはULCAの設計を参考に、CatapultサーバとAria10を組み合わせて3倍以上の性能向上を実現している。




