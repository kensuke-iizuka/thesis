\chapter{序論}
\section{本研究の背景}
近年、人工知能は、単なる情報サービスにとどまらず、医療、自動車、翻訳、その他多くの分野にまたがって活発に利用されている。
たとえば、いくつかの市販のカメラには、機械学習による画像識別によって、撮影された画像から誰が映っているかを判断し、人物ごとに分類しタグ付けを行う機能が組み込まれている。
このようなカメラやマイクを利用した身近な機械学習技術が普及する一方で、高度なデータ処理を逐次行うために多くの電力資源が消費されている。
機械学習技術の多くは、高度なビッグデータ処理が前提となっており、高性能化のためにも莫大な計算資源が必要となる。

この問題を解決するために、国立研究開発法人 新エネルギー・産業技術開発機構(NEDO)は2016年に
「省電力AIエンジンと異種エンジン統合クラウドによる人工知能プラットフォーム」のプロジェクトをスタートさせた。
これは、省電力GPUやセンサーを搭載した小規模システムのエッジ側と、FPGAやGPUを結合した大規模システムのクラウド側のふたつを開発し、
これらをタスクに合わせて効率的に運用することで電力性能の向上を目指すものである。

さて、このプロジェクトのクラウド側にあたる、FPGA、GPU、メモリなどを組み合わせた大規模システムFiC(Flow-in-Clowd)の開発が始まった。
これでは、FPGA搭載の高性能スイッチノードを中心とした大規模計算システムを構築し、特に、小規模システムのエッジ側では処理しきれない
機械学習の学習の処理の高性能化・低電力化を目的としている。

また、FiCシステムでは、初期のシステムソフトウェア開発用テストベッドとして、FPGAに多数の高速リンクを接続したFiC-SW1ボードを利用することを計画している。
このFiC-SW1ボードはFPGAとしてXilinx社の Kintex Ultrascale XCKU095を採用し、8Gbps全二重の光リンクとSTDM(Static Time Divi Time Division Multiplexing)による
サーキットスイッチングネットワークによってボード間を接続するという特徴がある。

本稿では、今後のFiCシステム開発のために障壁となる課題を発見するために、このFiC-SW1ボードの予備評価を行った。
そのために、このボード上に、画像識別において機械学習技術の中核となっている畳み込みニューラルネットワーク(CNN : Convolutional Neural Network)のうち、
ベンチマークとしてカラー入力画像を1000のカテゴリに分類するAlexNet \cite{alexnet}を実装し、FiC-SW1ボードの計算性能・転送性能をシミュレーションにより計測した。

\section{本稿の構成}
本稿の構成を示す。2章では本研究の要となるFiCシステムとFiC-SW1ボードについて紹介し、3章では、今回ベンチマークとして実装するCNNについて解説を行う。
4章では本研究に関連の深い研究を紹介する。5章では本研究の目的と課題を明らかにする。6章では、CNNをFiC上に実装するために、CNNの並列化を検討する。
7章では今回実装を行う畳み込みアクセラレータの構成を解説し、8章ではそれをFiC-SW1ボード上に実装した際のシミュレーションによる評価を行う。
9章では本研究の結論と今後の課題について述べる。