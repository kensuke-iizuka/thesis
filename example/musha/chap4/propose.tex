\chapter{目的}
本稿は、FiC-SW1ボード上のFPGAにCNNの畳み込み層の処理をする並列畳み込みアクセラレータを実装し、
その計算時間と通信時間の割合をシミュレーションによって測定することが目的である。そのためには、
以下の2つの課題に取り組まなくてはならない。畳み込み層の並列化と相互通信の検討については第6章、
畳み込みアクセラレータの実装については第7章で扱う。

\begin{description}
 \item[畳み込み層の並列化と相互通信の検討]\mbox{}\\ 
	    大規模な畳み込み演算を複数のFiC-SW1ボード上で処理するために最適なタスク分割を行う。
	    また、畳み込み演算を分割処理するにあたって相互通信が必要になる。その相互通信量と時間コストを静的に計算する。
 \item[畳み込みアクセラレータの実装]\mbox{}\\
	    分割された畳み込み演算を高速に行うために、FiC-SW1ボード上のFPGAに畳み込みアクセラレータの実装を行う。
\end{description}



