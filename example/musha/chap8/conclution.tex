\chapter{結論}
FiC(Flow-in-Cloud)システムは、NEDO人工知能プロジェクトにおいて、FPGAノード、GPUノード、メモリノードなどの異種ノードを多数組み合わせた大規模計算システムである。
このFiCシステム上で、スイッチノードしての役割を持つFiC-SW1は高速リンクを多数接続したFPGAボードである。
本稿では、このFiCシステム上でスイッチノードとしての役割と、初期のソフトウェア開発用テストベッドの役割を持つFiC-SW1をボードの計算性能と転送性能の
予備評価を行う目的で、大規模CNNのAlexNetをベンチマークとして、畳み込み演算の並列性について検討し、FIC-SW1上のFPGAに並列に畳み込み演算を行う畳み込みアクセラレータを実装した。

その結果、最適化を施していないアクセラレータでも一般的なCPUに比べて658倍高速化することに成功した。また、
通信時間に対する計算時間の比率は、2倍から10倍、最大で26倍になることがわかった。
ただし、通信時間の算出ではヘッダや通信遅延などを考慮していないため、実際の比率はこれより小さくなると予測される。

以下に今後の課題を述べる。本稿では、並列化・高速化の対象をCNNの畳み込み層の畳み込み演算のみに限って検証を行ったが、
CNNのプーリング層、正規化層、識別層についてもどのように並列化・高速化を行うか検討しなければならない。
また、今回実装した畳み込みアクセラレータは比較的単純な構造をしており、ダブルバッファリングなどの手法によってさらなる低リソース化・
高速化できる余地が残っている。アクセラレータを単純に5つ実装するにはリソースが足りないため、アクセラレータの改良やFPGA内でリソース共有を
行うことも検討する必要もある。
