\chapter{まとめと今後の課題}
{
\label{chap:conclusion}
\section{結論}
\label{sec:conclusion}
本研究ではNEDOが開発している大規模人工知能計算基盤FiCにおいて
スイッチノードとしての役割を担うマルチFPGAシステムが,スイッチとしての機能だけでなく,深層学習アクセラレータとして
利用できるのかをGoogLeNetの高速化とともに検討し実装,評価を行った.
実装はGoogLeNetの特徴に注目し,Inception層の並列性,畳込み演算の並列性を利用した並列化を検討した.
Inception層の各計算スレッドにそれぞれ,1枚ずつFPGAボードを割り当て,畳み込み演算器,Maxpooling演算器について
それぞれ処理の高速化を図り,Inception(3a)層をベンチマークとしてシュミレーションを行い評価をした.
シュミレーション結果として性能比で最適化をしないCPUに対して〇〇倍,最適化を図ったCPUに対して〇〇倍の高速化を実現した.

\section{今後の課題}
\label{sec:future}
本論文では基本的な高速化の手法のみを提案し,実装を行った.
シュミレーション結果ではあるがCPUに比べて高速化できたことから,FiC-SWで構築される
マルチFPGAシステムはCNNアクセラレータとして,活用が期待できる.
しかし,Inception層の各スレッドのロードバランスの検討やマルチFPGAをどのように繋いで,計算に適したネットワークを構築するか,
また畳み込み演算における入出力値分割の最適化など更なる性能向上,今回は計測しなかった電力効率の向上などを目指し,検討をしていく予定である.
}
