\chapter{実装}
{
\label{chap:implement}
本章では\ref{chap:parallel}章の並列化に対する検討を元にして、FPGAに実装するアクセラレータについて解説する。
\section{スレッド並列化の実装}
\label{sec:thread_impl}
図\ref{fig:para_inception}で示した4つの計算スレッドについて、各スレッドにそれぞれ1枚のFPGAで演算を行うようにする。

\section{スレッド内モジュールの実装}
\label{sec:module_impl}
FiCのマルチFPGAシステムでは、流れてきたデータをそれぞれのFPGAボードでの処理を終えて、次のFPGAに出力を渡す、というフローが想定される。
そのため各ボードでは特定の演算処理のみを実行すればよいので\cite{optimized}のように複数のサイズの畳み込みを行うことは考えずに、
それぞれの層に適した入出力サイズのモジュールを実装することが可能である。また特定の層の重みフィルタのみをBRAMに保存して読み出し演算を行う。
各スレッドのモジュールはまず、ブロードキャストされる同一サイズの入力値を受け取る。その後、BRAMの重みフィルタとの畳み込み演算を行う。
図\ref{fig:thread_module}にモジュールの模式図を示す。
\begin{figure}[h]
    \centering
    \includegraphics[width=12cm]{./chap6/fig/thread_module.pdf}
    \caption{モジュールの模式図}
    \label{fig:thread_module}
\end{figure}
入出力のインターフェイスにはXilinxが提供するAXI4-Streamを利用する。
AXI4-Streamはデータストリーム用のインターフェイスである。これにより入出力データをストリーム形式で送受信できる。
図\ref{fig:thread_module}に示すように入出力、重みフィルタにダブルバッファを設けることで演算に必要なデータを受け取ったら、
演算を行うと同時にもう片方のバッファで次のデータの書き込みが可能となる。

畳み込み演算はPEで行う。図\ref{fig:para_inception}で示した4つの計算スレッドのうちThread2、Thread3、Thread4については畳み込みやpooling処理
が連続して行われているので図\ref{fig:thread_module}が2つ接続するようなモジュールとなる。ストリーム処理を行い、パイプライン化するためにモジュール間には
バッファを設ける。

\section{畳み込み演算器の実装}
\label{sec:conv_impl}
\ref{sec:module_impl}節で述べたモジュールの模式図内のPE部の設計は\cite{optimized}を参考にした。
ここでは\ref{code:conv}に示す、畳み込み演算のCコードをループタイリングによって分割することで並列化することを考える。

\begin{itembox}[1]{畳み込み演算のCコード}
    \label{code:conv}
    \begin{verbatim}
    for (int r = 0; r < R; r++)
      for (int c = 0; c < C; c++)
        for (int to = 0; to < M; to++)
          for (int ti = 0; ti < N; ti++)
            for (int i = 0; i < K; i++)
              for (int j = 0; j < K; j++)
                output[to][r][c] +=
                  input[ti][S*r+i][S*c+j] *
                  weight[to][ti][i][j];
    \end{verbatim}
\end{itembox}

\begin{itembox}[1]{出力特徴マップチャンネルで並列化された畳み込み演算のCコード}
    \label{code:conv_tile}
    \begin{verbatim}
    for (int r = 0; r < R; r++)
      for (int c = 0; c < C; c++)
        for (int to = 0; to < M; to += Tm) //Tmごとに出力を分割
          for (int ti = 0; ti < N; ti++)
            for (int i = 0; i < K; i++)
              for (int j = 0; j < K; j++)
                output[to][r][c] +=
                  input[ti][S*r+i][S*c+j] *
                  weight[to][ti][i][j];
    \end{verbatim}
\end{itembox}

\ref{code:conv_tile}がループタイリングを行った疑似コードである。$T_m$が\ref{sec:conv_para}節で説明した、出力値による分割である.
この疑似コードに倣って演算モジュールを作ると図\ref{fig:conv_pe}のような演算器を設計できる。

\begin{figure}[h]
    \centering
    \includegraphics[width=12cm]{./chap6/fig/ucla_pe.png}
    \caption{畳み込み演算器の模式図}
    \label{fig:conv_pe}
\end{figure}

この演算器は複数の積和演算器を並列に並べたものとなっている.この演算器は$T_n$個の入力値と$T_n \times T_m$個の重みから
$T_m$個の出力値を得る.
\section{maxpooling演算器の実装}
\label{sec:max_impl}
maxpooling処理は式\ref{eq:pool}で示された最大値を取る演算を行う。
カーネルサイズの領域を入力特徴マップから抽出し、その領域内での最大値を取得し出力特徴マップに書き込むという処理を行う。
領域内の各要素について逐次的に前後の値を比較し、出力値の候補を探していく処理となるが、本研究では高速化を図るため、トーナメント方式に
値を比較する演算器を実装した。その模式図をmaxpooling処理の具体例とともに図\ref{fig:maxpool_pe}にモジュールの模式図を示す。
\begin{figure}[h]
  \centering
  \includegraphics[width=12cm]{./chap6/fig/maxpool_pe.pdf}
  \caption{モジュールの模式図}
  \label{fig:maxpool_pe}
\end{figure}
}

