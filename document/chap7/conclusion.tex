\chapter{まとめと今後の課題}
{
\label{chap:conclusion}
\section{結論}
\label{sec:conclusion}
VPCMAにおいて、実行するアプリケーションと要求性能に応じて電力を最小化するパイプライン段数とボディバイアス電圧を決定する手法を検討した。トレードオフの複雑さから単純に計算することが困難であったため探索はブルートフォースで行った。4つの24bit画像処理アプリケーションを実行するシミュレーションを行い評価を行った。

パイプライン段数に着目するとVPCMAおいてはアプリケーションや要求性能によらず全段にパイプライン分割した場合がもっとも電力が小さいとわかった。この結果はVPCMA特有のものであり、ブルートフォース探索によってによって明らかとなったアーキテクチャの特徴とも言える。このように本手法ではアーキテクチャに依存せずに電力を最小化するパイプライン段数を求めることができた。

ボディバイアス制御の粒度を行単位とすることによる効果をボディバイアス制御をしない場合、制御の粒度をPE\_ARRAY全体とした場合との比較を行うことにより検討した。
ボディバイアス制御をしない場合と比べて平均約46\%の電力削減率が得られた。一方で一律制御と比べると電力削減率は平均0.80\%であった。しかし、要求性能が高くなるにつれて行単位の制御では一律制御に対して高い電力削減率を示し、その効果はアプリケーションと要求性能に依存することがわかった。

\section{今後の課題}
\label{sec:future}

本論文で検討した手法にはブルートフォース探索を採用していることはすでに述べた。そのため、パイプライン分割のパターンを制限してもパイプライン段数とボディバイアス電圧を決定するのに時間がかかっていた。このようにパイプライン段数は粗粒度に探索を行っているため本研究では想定していないパイプライン分割のパターンの方がより小さい電力を示す可能性がある。ただし、本研究で示されたように電力を最小化する段数は使用している行数の付近にあるはずであり、探索はその周辺を行えばよい。よって、本論文で提案した手法のアルゴリズムには改良の余地がある。例えば遺伝的アルゴリズムなどを代表とする進化的計算アルゴリズムを採用し得られる結果と本研究のブルートフォース探索の結果を比較することにより、この手法に適するアルゴリズムを検討する必要があると考えられる。

アプリケーション依存性があるということは、どのPEにどの演算をマッピングするかに依存するということである。本研究ではアプリケーションのマッピングは固定したままでパイプライン
段数とボディバイアス電圧を変化させて電力を見積もり最小のものを探索していた。しかし、各々のパイプライン段数、ボディバイアス電圧に対する適切な演算のマッピングは異なる可能性がある。したがって、演算のマッピングを変化させた場合の影響についても評価を行う必要があると考えられる。
}
