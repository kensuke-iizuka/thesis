\chapter{評価}
{
\label{chap:eval}

\section{評価環境}
\label{sec:eval_env}
実装はVivado HLS(2017.3)を用いて実装を行った。対象となるFPGAボードは\ref{chap:ficsw}章の\ref{sec:about_ficsw}節でも言及した
Xilinx社のKintex Ultrascale XCKU095-FFVB2104である。動作周波数は100MHzとした。
Vivado HLS上でのシュミレーションの実行サイクル数から実行時間を計算した。
比較対象となるCPUはIntel(R) Xeon(R) CPU E5-2667 0(@2.90GHz)を用いて、g++4.4.7でO3最適化とOpenMPを使って
コンパイルしたアプリケーションの実行時間を計測した。

\section{リソース使用量と割合}
\label{sec:resource_util}
表\ref{table:resource_util}に各スレッドモジュールのFPGA合成結果のリソース使用率を示す。

\begin{table}[p]
    \begin{center}
    \caption{各スレッドにおけるFPGAボードのリソース使用量とその割合 []内は\%}
    \label{table:resource_util}
    \begin{tabular}{|c|c|c|c|c|} \hline
    \multicolumn{1}{|c|}{Thread} & \multicolumn{1}{|c|}{BRAM\_18K} & \multicolumn{1}{|c|}{DSP48} & \multicolumn{1}{|c|}{FF} & \multicolumn{1}{|c|}{LUT} \\ \hline \hline
    Thread1       & 10[1] & 10[1] & 10[1] & 10[1] \\ \hline
    Thread2       & 10[1] & 10[1] & 10[1] & 10[1] \\ \hline
    Thread3       & 10[1] & 10[1] & 10[1] & 10[1] \\ \hline
    Thread4       & 10[1] & 10[1] & 10[1] & 10[1] \\ \hline
    \end{tabular}
    \end{center}
\end{table}


\section{実行時間の比較}
\label{sec:resource_util}

表\ref{table:exec_time}に各スレッドモジュールのFPGA実行時間とCPUでの実行時間を示す。
また全ての演算を行った際の合計の実行時間も示す。

\begin{table}[p]
    \begin{center}
    \caption{各スレッドにおけるFPGAボードのリソース使用量とその割合 []内は\%}
    \label{table:exec_time}
    \begin{tabular}{|c|c|c|c|c|} \hline
    \multicolumn{1}{|c|}{Thread} & \multicolumn{1}{|c|}{CPU} & \multicolumn{1}{|c|}{CPU -O3} & \multicolumn{1}{|c|}{CPU -O3 \& OpenMP} & \multicolumn{1}{|c|}{FPGA} \\ \hline \hline
    Thread1       & 10[1] & 10[1] & 10[1] & 10[1] \\ \hline
    Thread2       & 10[1] & 10[1] & 10[1] & 10[1] \\ \hline
    Thread3       & 10[1] & 10[1] & 10[1] & 10[1] \\ \hline
    Thread4       & 10[1] & 10[1] & 10[1] & 10[1] \\ \hline
    Total       & 10[1] & 10[1] & 10[1] & 10[1] \\ \hline
    \end{tabular}
    \end{center}
\end{table}
}
