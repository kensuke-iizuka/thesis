\chapter{関連研究}
{
\label{chap:survey}

\subsection{マルチFPGAによる通信}
複数のFPGAにより構築されるシステムはMicrosoft社の提供するBing検索エンジンに利用されているCatapultが提案されている。
このシステムはデータセンター内のCPUにFPGAアクセラレータが接続されているだけではなくFPGAで2次元トーラスのネットワークを作ってノード間の通信も担っている。
通信方式はパケット交換方式なので回線交換を利用するFiCシステムとはこの点で異なる。

\subsection{GoogLeNetのFPGAによる実装}

\subsubsection{CaffeフレームワークのFPGA拡張とWinograd変換による3x3畳み込み演算の高速化検討}
CNNのフレームワークとして有名なCaffeをFPGA設計でも利用できるようにOpenCLで拡張し、さらにWinograd変換を
利用したフィルタサイズが3x3の畳み込み演算の高速化を検討。CPUやGPUに比べて、かえって処理に時間がかかるという結果になってしまったが
フレームワークをFPGA設計にも導入することでよりCNNアクセラレータの開発コストが下がり、将来の研究に役立つであろうとされている
\subsubsection{リソース分割によるCNNアクセラレータの効率最大化に関する研究}
chen zhangのアクセラレータなど先行研究の多くのFPGAを用いたCNNアクセラレータhは、
畳み込み演算処理装置(CLP: Convolutional Layer Processor)を1つしか持たない設計をしていることが多い。しかし複数の畳み込み層で
このCLP単体を使いまわすと各層の入出力のサイズの違いにより、リソース使用量が低下してしまったり、何度も使いまわさなければいけなくなってしまう。
この研究では複数のCLPを実装することでリソース使用率を最大化することを目指し、ネットワークのサイズによる最適化を行った、
その結果、AlexNetでは3.8x, squeezenetやGoogLeNetではそれぞれ2.2x, 2.0xを達成した
}