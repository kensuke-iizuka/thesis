\chapter{パイプライン型CGRA}
{
\label{chap:survey}

本章では本研究の電力改善手法が適用できるパイプライン型のCGRAアーキテクチャを説明する.本論文でパイプライン型CGRAと呼ぶアーキテクチャは以下のような特徴を持つアーキテクチャである.

\begin{itemize}
\item 単純なデータフローのみを扱う
\item パイプライン的な動作が可能である
\end{itemize}

PE間の相互接続のパターンを制限することにより,データフローが特定の方向に流れるように単純化されている.そのためPEアレイと外部にあるデータメモリとの間の入出力も単純化されている.このことによる入出力の柔軟性低下に対処するためにデータメモリとPEアレイとの間にクロスバのような機構が備わっていることが多い.明示的なパイプラインレジスタが必ずしも挿入されているわけではなくPEの内部に中間データを保持するレジスタが備えられていることもある.その場合でもデータフローがPEアレイの列方向となるように単純化されている場合,PEアレイの行をパイプラインステージとみなすことができる.マッピングされる演算によって各パイプラインステージには遅延のばらつきが発生する.パイプラインステージ単位でボディバイアス制御を行うことでこのばらつきを調整することができる.さらに,要求される性能が低い場合はパイプラインステージでの遅延が増加するような構成も許される.ゆえに,演算のマッピングや相互接続などの構成情報を変化させてパイプラインステージにおける遅延を調節することと,同時ににボディバイアス電圧を考慮することが可能である.本研究で対象としたVPCMAではパイプラインステージの遅延を調節する手法としてパイプライン段数を変化させている.VPCMA以外のアーキテクチャでパイプライン型CGRAに分類されるアーキテクチャはPipeRench\cite{piperench}, S5 Engine\cite{s5},XPP\cite{xpp},DT-CGRA\cite{DT-CGRA}
が提案されている.これらのアーキテクチャの概要を説明する.

\section{PipeRench}
\label{sec:PipeRench}
PipeRenchアーキテクチャはPEを16個水平方向に並べて1stripeと呼んでいる.16stripeのチップが実装されている.データはstripeに対して垂直に流れていくためstripeがパイプラインの1ステージと考えることができる.PipeRenchは物理的には16stripeしか存在しないが仮想的に256stripeまで利用できるのが特徴である.演算が終了したstripeを次のステージのstripeとして再構成することで仮想化を実現している.本手法を適用すれば使用するstripeの数とstripe単位のボディバイアス制御によって電力を最小化することができる.


\section{s5 Engine}
\label{sec:s5}
S5 Engineはと呼ばれるアーキテクチャは主に5つの機能ブロックから構成されている.
\begin{enumerate}
\item 命令フェッチとデコードを行うユニット
\item ロード,ストアを行うユニット
\item 整数演算ユニット
\item 浮動小数点演算ユニット
\item 拡張ユニット
	\begin{itemize}
	\item アレイ状のALU
	\item アレイ状の乗算器
	\end{itemize}
\end{enumerate}

上の4つは一般的なプロセッサと同様の動作をする.拡張ユニットはアレイ状のALUと乗算器を持つ.拡張ユニットでの動作はチップ製造後にプログラマ及びコンパイラが指定可能である.上記ユニットは固定的な5段のパイプライン化がされているのに対して,拡張ユニットにおけるパイプラインの段数は可変的である.拡張ユニットの動作周波数はプログラム可能であり要求性能に応じてプログラマが指定することが可能である.拡張ユニットにおいてボディバイアス制御をできるようにすることで本手法を適用し電力を最小化するボディバイアス電圧とパイプライン段数を決定することができる.

\section{XPP}
\label{sec:XPP}
マルチメディア処理,通信,グラフィックス処理において見られる高い並列度のストリーム処理に適したアーキテクチャである.構成要素は以下の4つに分けることができる.構成要素の一つであるPAEとはProcessing Array Elementの略である.さらにALU-PAEは3つに細分化できる.
\begin{enumerate}
\item ALU-PAE
	\begin{itemize}
	\item ALU
	\item FREG
	\item BREG
	\end{itemize}
\item RAM-PAE
\item IO-Element
\item Configuration manager
\end{enumerate}

ALU-PAEに含まれるALUでは乗算や加算,比較,ソート,シフト演算などのDSP処理で見られる典型的な演算を行う.FREGでは垂直方向下向きのデータフローを管理する.さらにマルチプレクサやスワップなどのデータフローの制御用の演算が可能である.BREGでは垂直方向上向きのデータフローを管理する.さらに,加算やバレルシフト,正規化を行うことが可能である.
RAM-PAEは中間データを保持するための要素であり,構成情報を変化させることでFIFOとして動作させることも可能である.IO-ElementはPAEのアレイと外部のRAMとのやりとりを行う.Configuration Managerは各PAEの構成を管理する.

XPPのアレイはデータ幅や各PAEの個数などがパラメータ化されたソフトコアとして提供されている.そのため様々な設計が試されている.その一例はALU-PAEが64個,RAM-PAEが16個,IO-Elementが4個の設計がある.これは8x8のALU-PAEのアレイの左右にRAM-PAEが8個ずつ配置されている.IO-Elementはこのアレイの上側と下側に配置されている.つまり,XPPのアレイへのデータは上側または下側から入力され,上側または下側から出力する単純なデータフローとなる.各PAEにはアレイ全体で同期を取るために入力部と出力部にレジスタがある.したがって,アレイの1行を1ステージとしたパイプラインとして動作していると考えることができる.

\section{DT-CGRA}
\label{sec:DT-CGRA}
DT−CGRAアーキテクチャは画像処理や機械学習向けのアクセラレータとして提案されている.構成要素を以下に示す.
\begin{enumerate}
\item CA(Computing Array)
	\begin{itemize}
	\item RC(Reconfigurable Cell)
	\item PRC
	\item IRC
	\end{itemize}
\item SBU(Stream buffer units)
\item クロスバ
\end{enumerate}

CAは複数列のRCと特別な計算を行うRCが1列で構成されている.特別な演算のできるRCはPRCとIRCがあり,PRCでは平方根や逆数の計算,IRCでは補間法で用いられる計算を行うことができる.RC間で転送するデータ量を減らすためにMapReduceの手法が用いられている.各RCには5つのPEを持っている.このうち3つをmap処理に,2つをreduce処理に用いる.ある処理をRCにマッピングするとき,map処理で必要なPEの数が3つ以上の場合隣接したRCを用いることで対応することが可能である.一方で,map処理に用いるPEの数が3つ未満の場合隣接するRCに使用させることが可能である.SBUでは外部のデータメモリから転送されたCAへの入力データやCAからの出力データを保持する.クロスバはSBUとCAとの間に配置されていて,データ転送の柔軟性を高める.

画像処理や機械学習などのアルゴリズムではデータの局所性が強く,異なるデータに対して同じ計算をすることが多い.このことに着目しDT-CGRAではCAの構成は静的に決定され,動的な再構成が行われない.一方でSBUからCAへ入力されるデータ数,入力するCAの行などは動的に再構成される.固定的な計算のマッピングに対してデータフローを動的に制御することで柔軟性を得ている.

演算処理部のマッピングを静的な構成にする点はVPCMAのコンセプトに近い.また,データ転送をクロスバを用いることで工夫している点はVPCMAにおいて\ref{subsec:micro_controller}節で説明するデータマニピュレータを使用している点に類似している.RC間の相互接続は他方向に伸びており単純なデータフローとは呼べない.しかし,RC内部が複数のPEで構成されること演算部の構成が静的であることからRCがパイプラインの1ステージとみなすことができる.同じアルゴリズムをマッピングするとしてもmap,reduce処理の組をいくつのRCを用いるかによってパイプラインステージにおける遅延を調節することができる.RC単位のボディバイアス制御を行えば本手法を適用し電力最小化をすることが可能である.

}