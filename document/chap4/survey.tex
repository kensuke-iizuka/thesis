\chapter{関連研究}
{
\label{chap:survey}

\subsection{マルチFPGAによる通信}
複数のFPGAにより構築されるシステムはMicrosoft社の提供するBing検索エンジンに利用されているCatapult\cite{catapult1st}が提案されている.
このシステムはデータセンター内のCPUにFPGAアクセラレータが接続されているだけではなくFPGAで2次元トーラスのネットワークを作ってノード間の通信も担っている.
通信方式はパケット交換方式なので回線交換を利用するFiCシステムとはこの点で異なる.

\subsection{GoogLeNetのFPGAによる実装}

\subsubsection{CaffeフレームワークのFPGA拡張とWinograd変換による3x3畳み込み演算の高速化検討}
CNNのフレームワークとして有名なCaffeをFPGA設計でも利用できるようにOpenCLで拡張し,さらにWinograd変換を
利用したフィルタサイズが3x3の畳み込み演算の高速化を検討し実装した\cite{caffeinated}.CPUやGPUに比べて,かえって処理に時間がかかるという結果になってしまったが
フレームワークをFPGA設計にも導入することでよりCNNアクセラレータの開発コストが下がり,将来の研究に役立つであろうとされている.
\subsubsection{リソース分割によるCNNアクセラレータの効率最大化に関する研究}
chen zhangらによるアクセラレータ\cite{optimized}など先行研究の多くのFPGAを用いたCNNアクセラレータは,
畳み込み演算処理装置(CLP: Convolutional Layer Processor)を1つしか持たない設計をしていることが多い.しかし複数の畳み込み層で
このCLP単体を使いまわすと各層の入出力のサイズの違いにより,リソース使用量が低下してしまったり,何度も使いまわさなければいけなくなってしまう.
この研究では複数のCLPを実装することでリソース使用率を最大化することを目指し,ネットワークのサイズによる最適化を行った\cite{max}
その結果,AlexNet\cite{alexnet}では3.8x, squeezenet\cite{googlenet}やGoogLeNet\cite{googlenet}ではそれぞれ2.2x, 2.0xを達成した
}