\chapter{序論}
{
\label{chap:introduction}
\section{本研究の背景}
\label{sec:backgroud}
ウェアラブルデバイスを始めとする高機能で小型なデバイスが求められている。これらのデバイスはバッテリーを電力源としていることが多く、一定以上の性能を低消費電力で実現することが要求されている。この要求を満たすためにエネルギー効率の高いアクセラレータを利用することが検討されている。特にそのエネルギー効率の高さから粗粒度再構成可能アーキテクチャ(CGRA: Coarse-Grained Reconfigurable Architecture)の一種である動的再構成可能プロセッサ(DRP: Dynamic Reconfiguable Processor)が注目されている。このDRPはPE(Processing Element)と呼ばれる小さな演算処理部分をアレイ状に多数持っており、PE間の相互接続や演算の種類などをクロックに同期して切り替えることで動的に構成を変化させることができる。汎用的なプロセッサとは異なり命令のフェッチなどが不要であるため低い周波数、小さい電力で高い性能を得ることができる。

一方でDRPはPEアレイへのクロック分配や中間データを保持するレジスタ、動的な再構成による電力が大きな割合を占めている問題があった。我々の研究室で開発した動的再構成可能プロセッサMuCCRA-3において動的再構成が全体の25\%、クロックツリーでは全体の15\%の消費電力を占めていた。\cite{muccra}

これらの電力を削減するためこれまでに高性能で低電力なアクセラレータのためのアーキテクチャとしてCMA(Cool Mega-Array)が提案されている。\cite{cma_micro}CMAはPEアレイからクロックツリーを取り除き、動的な再構成を行わないことで消費電力の削減を図っている。
% これによって柔軟性が低下するがマイクロコントローラによるメモリの読み書きを工夫することで対処している。
したがってPEアレイにおける柔軟性が低下してしまうが、マイクロコントローラによるメモリ入出力を工夫することで対応することができる。
さらにCMAを低電力化するためにFully Depleted Sicilon On Insulator(FD-SOI) の一種である Silicon On Thin Buried oxide(SOTB)が利用されている。これまでにCMA-SOTB\cite{cma-sotb}、CMA-SOTB2\cite{cma-sotb2}、CMA-CUBE-SOTB(CC-SOTB)\cite{cc-sotb}が開発された。SOTBプロセスの特徴はボディバイアス電圧を印加することで遅延とリーク電力のトレードオフさせることができる点であり、CMAの性能と電力の最適化を行うことが可能となった。

CMA-SOTB2以前のCMAではPEアレイにデータを入力すると演算結果が出力されるまで新たにデータを入力することができなかった。しかし、演算の終盤においては序盤に使用されたPEは使用されないことがほとんどであった。そこでCC-SOTBでは複数の入力データに対して演算をオーバーラップして実行可能にするためのウェーブパイプラインモードが実装された。ところが、ウェーブパイプラインモードでは前段のデータが後段のデータと衝突する可能性があった。また、PE\_ARRAYが大きな組み合わせ回路で構成されているためグリッチによる消費電力が問題となっていた。こうした課題に対処するために、安全に動作可能なパイプラインの実装とその評価が行われた。パイプラインの段数は可変的でありアプリケーションに応じてパイプライン段数を変化させることが可能な可変パイプラインである。CC-SOTBと比べ平均77\%の性能向上と最大で1461MOPS\footnote{Million Operation Per Second}/mWという非常に高いエネルギー効率を達成できることが報告されている。\cite{vpcma}しかし、これらの評価にはSOTBプロセス特有の利点であるボディバイアス制御による電力削減を行っていなかった。そこで本研究では提案された可変パイプラインのCMA(VPCMA: Variable Pipelined CMA)に対してパイプライン段数とボディバイアス制御を同時に行うことでさらなる電力効率の改善を検討する。

\section{研究目的}
\label{sec:purpose}

本研究の目的は低電力アクセラレータVPCMAにおいて、電力を最小化するパイプラインの段数の変化とボディバイアス電圧の制御を同時に行うことによる電力削減の効果を明らかにすることである。さらにパイプライン段数とボディバイアス電圧を決定する手法を確立し、類似するアーキテクチャへ適用可能かを検討する。

\section{本論文の構成}
\label{sed:composition}
\ref{chap:vlsi}章ではCMOSプロセスを利用したディジタル回路における消費電力の見積もり方法とSOTB技術の概要を説明する。\ref{chap:survey}章では本研究で検討するパイプライン段数選択とボディバイアス電圧制御が適用できるパイプライン型のCGRAを紹介する。\ref{chap:vpcma}章では本研究で検討する手法を適用するアーキテクチャのVPCMAの概要を説明する。\ref{chap:proposal}章では本研究の電力最小化の手法に関して説明する。\ref{chap:eval}章では本研究の予備評価と電力最小化の評価を行う。 \ref{chap:conclusion}では本論文の結論を述べる。


}

