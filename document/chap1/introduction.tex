\chapter{序論}
{
  \label{chap:introducion}

  \section{本研究の背景}
  \label{sec:backgroud}
  人工知能は爆発的な普及を見せていて,自動運転やスマートスピーカー,スマートフォン向けアプリケーションなど様々なシステムに取り込まれている.
  人工知能は機械学習が主な技術として実現されているが,機械学習には大量のデータとそれを取り扱う大量の演算が必要となる.
  そのため人工知能のさらなる普及,発展にはその計算基盤が必要である
  特に画像認識や物体検出など人工知能の発展に不可欠な分野で活躍する畳み込みニューラルネットワーク(CNN: Convolutional Neural Network)は
  計算の特性から汎用CPUでは効率よく演算処理ができない.
  インテルやNVDIAなど大手半導体メーカーを始めとしてGoogleやMicrsoftなど人工知能サービスを提供する大企業も
  人工知能向け専用アクセラレータの開発に力を注いでいる.
  各社,研究機関はGPU,ASIC,FPGAなど様々なハードウェア,手法で高速化を図る.
  日本でも国立研究開発法人新エネルギー・産業技術開発機構(NEDO)が
  「省電力AIエンジンと異種エンジン統合クラウドによる人工知能プラットフォーム」プロジェクトで
  エッジ側では推論処理(カメラに何が映っているかを特定するなどの処理)の省電力化を目指したエンジンを,
  クラウド側では複数のFPGA,GPU,メモリなどの異種ノードを多数接続した大規模人工知能計算基盤Flow-in-Clowd(FiC)を開発している.
  このFiCはデータセンターなどに導入されるクラウドシステムとしてIoTのセンサから取得したデータなどを学習処理(教師データから推論モデルの構築)を行う.
  FiCシステムの主演算装置となる複数のGPUを複数のFPGAのスイッチノードに接続し,
  高速通信を行う.高機能スイッチノードととなるマルチFPGAは多数の高速リンクが接続され,FiCの高速通信のスイッチングの役割を担う.

  FPGAは電力効率のよさ,開発周期の短さ,再構成可能であることなどから注目され推論処理のアクセラレータとしても様々な研究が行われている.
  GPUは学習処理は非常に高い性能を示すが,推論においては効率が良いとはいえず,また消費電力も大きい.
  ,またASIC開発は非常に開発に時間がかかり,たとえ開発に成功したとしてもその莫大な開発コストに見合う大きな市場を探すのは困難である.
  以上の理由からFiCを構成するマルチFPGAシステムはスイッチノードという役割に加え,AIエンジンとしての役割を担うことも期待されている.
  そこで本研究ではマルチFPGAシステムの試作ボードであるFiC-SW1を複数枚用いて,CNNのモデルであるGoogLeNetを実装し,評価を取った.

  \section{研究目的}
  \label{sec:purpose}
  本研究の目的はFiCに搭載されているマルチFPGAシステムの試作ボード,FiC-SWを複数枚用いてGoogLeNetを実装し,
  評価を取ることで既存研究や汎用CPU,GPUと比較してどの程度の性能向上,電力効率となるかを調べ,実際にクラウドシステムのAIエンジンとして
  実用的かどうか調べることである.

  \section{本論文の構成}
  \label{sec:composition}
  \ref{chap:googlenet}章では実装対象であるGoogLeNetと畳込みニューラルネットワークの概要を説明する.
  \ref{chap:ficsw}章ではFiCプロジェクトの概要と本研究で用いるマルチFPGAシステムを紹介する.
  \ref{chap:survey}章では本研究に関連する先行研究について説明する.
  \ref{chap:parallel}章ではGoogLeNetの並列化手法について説明する.
  \ref{chap:implement}章では\ref{chap:parallel}での並列化を考慮した実装方法について説明する. 
  \ref{chap:eval}章では本研究の評価を行う. 
  \ref{chap:conclusion}では本論文の結論を述べる.
}
