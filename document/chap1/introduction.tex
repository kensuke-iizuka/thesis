\chapter{序論}
{
    \label{chap:introducion}

    \section{本研究の背景}
    \label{sec:backgroud}
    人工知能と称される機械学習をベースにした技術は爆発的な普及を見せていて日夜、メディアで取り上げられるだけでなく、
    自動運転やスマートスピーカー、スマートフォン向けアプリケーションなど様々なシステムに取り込まれている。
    しかし、人工知能のさらなる普及にはその計算基盤が必要である。
    その中でも特に画像認識や物体検出などの分野で活躍する畳み込みニューラルネットワーク(CNN)はその計算の特性から
    汎用CPUでは効率よく演算処理ができない.インテルやNVDIAなど大手半導体メーカーを始めとしてGoogleやMicrsoftなども
    人工知能向け専用アクセラレータの開発に心血を注いでいる。
    各社,研究機関はGPU,ASIC,FPGAなど様々なデバイス,手法で高速化を図る.その中でもFPGAはその電力効率のよさ,開発周期の短さ,再構成可能であることから
    注目されている。
    日本でも国立研究開発法人新エネルギー・産業技術開発機構(NEDO)は「省電力AIエンジンと異種エンジン統合クラウドによる人工知能プラットフォーム」と銘打ったプロジェクトで
    複数のFPGA、GPU、メモリなどの異種ノードを多数接続した大規模計算基盤Flow-in-Clowd(FiC)を開発している。
    このシステムは車載などを想定したエッジとデータセンターなどで運用されるクラウドシステムからなる。クラウドシステムであるFiCは主演算装置となる複数のGPUを高速リンクで接続された
    複数のFPGAのスイッチノードにより接続し、
    複数のFPGAは高機能スイッチノードとして多数の高速リンクが接続され、FiCの高速通信のスイッチングの役割を担う。
    このマルチFPGAシステムは更にAIエンジンとしての役割も担う。

    \section{研究目的}
    \label{sec:purpose}
    本研究の目的はマルチFPGAシステム上にGoogLeNetを実装し、既存研究や汎用CPU、GPUに対して
    性能向上を目指すことである

    \section{本論文の構成}
    \label{sec:composition}
    \ref{chap:googlenet}章では実装対象であるGoogLeNetと畳込みニューラルネットワークの概要を説明する。
    \ref{chap:ficsw}章では本研究で用いるマルチFPGAシステムとそのプロジェクトの概要を紹介する。
    \ref{chap:survey}章では本研究に関連する先行研究について説明する。
    \ref{chap:paralell}章ではGoogLeNetの並列化手法について説明する。
    \ref{chap:implement}章では\ref{chap:paralell}での並列化を考慮した実装方法について説明する。 
    \ref{chap:eval}章では本研究の評価を行う。 
    \ref{chap:conclusion}では本論文の結論を述べる。
% \label{chap:introduction}
% \section{本研究の背景}
% \label{sec:backgroud}
% ウェアラブルデバイスを始めとする高機能で小型なデバイスが求められている。
% これらのデバイスはバッテリーを電力源としていることが多く、一定以上の性能を低消費電力で実現することが要求されている。
% この要求を満たすためにエネルギー効率の高いアクセラレータを利用することが検討されている。
% 特にそのエネルギー効率の高さから粗粒度再構成可能アーキテクチャ(CGRA: Coarse-Grained Reconfigurable Architecture)の一種である
% 動的再構成可能プロセッサ(DRP: Dynamic Reconfiguable Processor)が注目されている。
% このDRPはPE(Processing Element)と呼ばれる小さな演算処理部分をアレイ状に多数持っており、PE間の相互接続や演算の種類などをクロックに同期して切り替えることで動的に構成を変化させることができる。
% 汎用的なプロセッサとは異なり命令のフェッチなどが不要であるため低い周波数、小さい電力で高い性能を得ることができる。

% 一方でDRPはPEアレイへのクロック分配や中間データを保持するレジスタ、動的な再構成による電力が大きな割合を占めている問題があった。
% 我々の研究室で開発した動的再構成可能プロセッサMuCCRA-3において動的再構成が全体の25\%、クロックツリーでは全体の15\%の消費電力を占めていた。\cite{muccra}

% これらの電力を削減するためこれまでに高性能で低電力なアクセラレータのためのアーキテクチャとしてCMA(Cool Mega-Array)が提案されている。
% \cite{cma_micro}CMAはPEアレイからクロックツリーを取り除き、動的な再構成を行わないことで消費電力の削減を図っている。
% % これによって柔軟性が低下するがマイクロコントローラによるメモリの読み書きを工夫することで対処している。
% したがってPEアレイにおける柔軟性が低下してしまうが、マイクロコントローラによるメモリ入出力を工夫することで対応することができる。
% さらにCMAを低電力化するためにFully Depleted Sicilon On Insulator(FD-SOI) の一種である Silicon On Thin Buried oxide(SOTB)が利用されている。
% これまでにCMA-SOTB\cite{cma-sotb}、CMA-SOTB2\cite{cma-sotb2}、CMA-CUBE-SOTB(CC-SOTB)\cite{cc-sotb}が開発された。
% SOTBプロセスの特徴はボディバイアス電圧を印加することで遅延とリーク電力のトレードオフさせることができる点であり、CMAの性能と電力の最適化を行うことが可能となった。

% CMA-SOTB2以前のCMAではPEアレイにデータを入力すると演算結果が出力されるまで新たにデータを入力することができなかった。
% しかし、演算の終盤においては序盤に使用されたPEは使用されないことがほとんどであった。
% そこでCC-SOTBでは複数の入力データに対して演算をオーバーラップして実行可能にするためのウェーブパイプラインモードが実装された。
% ところが、ウェーブパイプラインモードでは前段のデータが後段のデータと衝突する可能性があった。
% また、PE\_ARRAYが大きな組み合わせ回路で構成されているためグリッチによる消費電力が問題となっていた。
% こうした課題に対処するために、安全に動作可能なパイプラインの実装とその評価が行われた。
% パイプラインの段数は可変的でありアプリケーションに応じてパイプライン段数を変化させることが可能な可変パイプラインである。
% CC-SOTBと比べ平均77\%の性能向上と最大で1461MOPS\footnote{Million Operation Per Second}/mWという非常に高いエネルギー効率を達成できることが報告されている。
% \cite{vpcma}しかし、これらの評価にはSOTBプロセス特有の利点であるボディバイアス制御による電力削減を行っていなかった。
% そこで本研究では提案された可変パイプラインのCMA(VPCMA: Variable Pipelined CMA)に対してパイプライン段数とボディバイアス制御を同時に行うことでさらなる電力効率の改善を検討する。




}

