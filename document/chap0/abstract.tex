\jabst{
    昨今,人工知能が最新技術のトレンドとして,様々なメディアに取り上げられ様々な製品に組み込まれている.
    特に画像認識や音声処理,物体検出などの分野で大きな貢献を果たしているニューラルネットワークは一躍注目されていて,研究開発が盛んに行われている.
    ニューラルネットワークの一種である畳み込み演算を主な計算とする畳み込みニューラルネットワーク(CNN)は精度向上のために計算量が増加する傾向にある.
    CNNの認識精度向上のためにはのためにはそれを支える高速化,電力性能向上が求められている.
    各半導体メーカや研究機関は専用のアクセラレータの開発に取り組んでいる.
    日本でも国立研究開発法人新エネルギー・産業技術開発機構(NEDO)は「省電力AIエンジンと異種エンジン統合クラウドによる人工知能プラットフォーム」と銘打ったプロジェクトで
    複数のFPGA,GPU,メモリなどの異種ノードを多数接続した大規模計算基盤Flow-in-Clowd(FiC)を開発している.
    複数のFPGAは高機能スイッチノードとして多数の高速リンクが接続され,FiCの高速通信のスイッチングの役割を担う.
    このマルチFPGAシステムは更にAIエンジンとしての役割も担う.
    本研究ではマルチFPGAシステムに2014年のILSVRCで最高精度をマークしたCNNモデルの1つであるGoogLeNetを実装し,
    性能でCPUの〇〇倍,GPUの〇〇倍を達成し電力効率でCPUの〇〇倍,GPUの〇〇倍を達成した.
}