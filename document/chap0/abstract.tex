\jabst{
    昨今,人工知能が最新技術のトレンドとして日夜,様々なメディアに取り上げられている.
    特に画像認識や音声処理,物体検出などの分野で大きな貢献を果たしているニューラルネットワークは一躍注目されていて,研究開発が盛んに行われている.
    畳み込み演算を主な計算とする畳み込みニューラルネットワーク(CNN)はその計算量の多さから,高速化,電力性能向上が求められている.
    各半導体メーカや研究機関は専用のアクセラレータの開発に心血を注いでいる.
    日本でも国立研究開発法人新エネルギー・産業技術開発機構(NEDO)は「省電力AIエンジンと異種エンジン統合クラウドによる人工知能プラットフォーム」と銘打ったプロジェクトで
    複数のFPGA,GPU,メモリなどの異種ノードを多数接続した大規模計算基盤Flow-in-Clowd(FiC)を開発している.
    複数のFPGAは高機能スイッチノードとして多数の高速リンクが接続され,FiCの高速通信のスイッチングの役割を担う.
    このマルチFPGAシステムは更にAIエンジンとしての役割も担う.
    本研究ではマルチFPGAシステムに2014年のILSVRCで最高精度をマークしたCNNモデルの1つであるGoogLeNetを実装し,
    性能でCPUの〇〇倍,GPUの〇〇倍を達成し電力効率でCPUの〇〇倍,GPUの〇〇倍を達成した.
}


% \jabst{
%     IoT時代においてウェアラブルデバイスなどのバッテリー駆動の高機能で小型なデバイスが求められている.
%     こうしたデバイスは一定以上の性能を低消費電力で実現することが要求される.
%     そのためエネルギー効率の高い粗粒度再構成可能アーキテクチャ(CGRA: Coarse-Grained Reconfigurable Architecture)が注目されている.
%     CGRAはPE(Processing Element)と呼ばれる小さな演算処理部をアレイ状に多数持っていて,PEの構成をチップ製造後に変えることで高い柔軟性を持っている.
%     これまでCGRAの一種であるCMA(Cool Mega-Array)アーキテクチャが提案されている.CMAは動的な再構成を行わないためPEアレイへのクロック分配,中間データのレジスタを廃し高い電力効率を実現している.
%     さらにCMAの1種であるVPCMA(Valiable Pipelined CMA)と呼ばれる可変パイプラインのCMAが提案されている.
%     VPCMAではパイプライン分割のための小さなオーバーヘッドに対して高い性能向上とエネルギー効率が報告されている.
%     しかし,これらの評価はパイプライン段数を変化させるだけに留まり,ボディバイアス制御によるリーク電力の制御を行っていなかった.
    
%     そこで本研究ではVPCMAに対してパイプライン段数とボディバイアス制御を同時に行うことでさらなる電力効率を改善する手法を検討した.
%     また,この手法ではPEアレイに対して一律のボディバイアス電圧を与えるのではなく,さらに小さな粒度で行単位でボディバイアス電圧を制御することによる効果を検討した.
%     まず初めに実行するアプリケーションと要求性能に対して電力を最小化するパイプライン段数とボディバイアス電圧を決定する手法を提案した.
%     4つの24bit画像処理アプリケーション(af,sf,sepia,gray)の実行をシミュレーションし評価を行った.
%     この手法を適用した結果,電力を最小化するパイプライン段数はPEアレイで使用している行の数と同数であることがわかり,アプリケーションのマッピングに依存することがわかった.
%     またPEアレイに対して行毎のボディバイアス制御を行うと,ゼロバイアス時と比べて平均約46\%の電力削減率を得た.
%     一方でPEアレイ全体で一律のボディバイアス制御をする場合と比べると電力削減率は平均約0.80\%であった.
%     }
